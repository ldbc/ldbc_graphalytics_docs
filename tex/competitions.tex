\chapter{\toolname-based Competitions} \label{chap:competitions}

\futureinversion{1.0}{Add examples for each type of competitions.} .

In this chapter, we describe how to participate in \toolname-based competitions. 
\toolname{} defines several {\it official competitions} (\sref{sec:competitions:overview}), which are open globally to everyone who satisfies the rules for participation. Each competition ranks the benchmark submission based on a competition method defined in \sref{sec:competitions}.




\section{Official \toolname{} Competitions}
\label{sec:competitions:overview}
Currently, \toolname{} defines two official competitions: (1)~the Global LDBC Competition and (2)~the Global \toolname{} Competition.


\subsection{The Global LDBC Competition} \label{sec:competitions:ldbc}
The Global LDBC Competition is maintained by LDBC, in particular by the Graphalytics Task Force. By the rules of the LDBC charter~\cite{ldbc_byelaws}, the competition method follows the single value-of-merit approach described in \sref{sec:competitions:single_value}, and focuses on two primary metrics: {\bf ``performance''} and {\bf ``cost-performance''}.

The competition reports the following list of metrics:
\begin{enumerate}
    \item (informative only) Full disclosure of the ``system under test'' (platform + environment).
    \item (informative only) {\it Target scale} of the benchmark.
    \item (informative only) {\it Date} of the benchmark execution.
	\item (flagship value-of-merit) {\it Performance metric}, as summarized from ``EVPS'' of all benchmarked jobs.
	\item (capability value-of-merit) {\it Cost-performance metric}, as summarized from ``PPP'' of all benchmarked jobs.
	\item (informative only) Three {\it performance metrics}, as summarized from $T_l$, $T_m$, and $T_p$ respectively.
\end{enumerate}

\begin{description}
    \item[Maintained by:] LDBC, \url{ldbcouncil.org}.
    \item[Audience:] The LDBC Competition accepts submissions from a global audience.
\end{description} 

\futureinversion{2.0}{Ratio metrics (scalability) and additional energy-related metrics will also be considered.} .



\subsection{The Global \toolname{} Competition} \label{sec:competitions:graphalytics}
The Global \toolname{} Competition is maintained by the \toolname team. The competition method follows the tournament-based approach described in \sref{sec:competitions:tournament}, and focuses on two primary scores: {\bf ``performance''} and {\bf ``cost-performance''}.

The Graphalytics consists of a number of {\it matches}, where each match represents a type of experiment that focuses on a specific performance characteristic that is common across all systems, for example, the EVPS of the BFS algorithm on a Datagen dataset. Each match consists of a set of instances, with the {\bf tournament score} being for each system the sum of {\bf instance scores} accumulated by the platform across all matches in which it participates.  Each {\it instance} is a head-to-head comparison between two systems, for example, comparing the EVPS of any algorithm-dataset for the pair (Giraph, GraphX): the winner receives 1 point, the loser 0 points, and a draw rewards each platform with 0.5 points each.


\begin{enumerate}
    \item (informative only) Full disclosure of the ``system under test'' (platform + environment).
    \item (informative only) {\it Target scale} of the benchmark.
    \item (informative only) {\it Date} of the benchmark execution.
	\item (ranking) {\it Performance score}, by comparing pair-wisely ``EVPS'' of all benchmarked jobs.
	\item (ranking) {\it Cost-performance score}, by comparing pair-wisely  ``PPP'' of all benchmarked jobs.
\end{enumerate}


\begin{description}
    \item[Maintained by:] \toolname, \url{graphalytics.org}.
    \item[Audience:] The Global \toolname{} Competition accepts submissions from a global audience.
\end{description} 



\futureinversion{2.0}{Ratio metrics (scalability) and additional energy-related metrics will also be considered.} .








\section{Competition Method} \label{sec:competitions}
Different competition methods have been developed to performance comparison in many application domains. Comparing multiple platforms across multiple performance metrics is not trivial. Two major approaches exist for this task: (i) creating a compound metric, typically by weighting the multiple metrics, and comparing multiple platforms using only this single-value-of-merit, and (ii) using a tournament format that allows for multiple participants (platforms) to be compared across multiple criteria. 

The former requires metrics to be easy to compare and compose, that is, to be normalized, to be similarly distributed, to have the same meaning of better (e.g., lower values), to be of importance universally recognized across the field of practice so that weights can be easily ascribed. The latter requires a good tournament format, which does not favor any of the participants, and which does not make participation cumbersome through a large set of rules.


\subsection{Single Value-of-merit Approach} \label{sec:competitions:single_value}
Where metrics (see \sref{sec:def:metrics}) are collected repeatedly, e.g., each combination of algorithm and dataset, a single value-of-merit can be summarized following the typical processes of benchmarking HPC systems~\cite{DBLP:conf/sc/HoeflerB15}:

\begin{itemize}
	\item For \textbf{Performance metrics,} the \emph{arithmetic mean} across all data.
	\item For \textbf{Throughput metrics,} because they are rate metrics, in two consecutive steps:
		\begin{enumerate}
			\item let $a$ be the \emph{arithmetic mean} of the performance metric (e.g., processing time) and $w$ be the (constant, total) workload (e.g., count of edges plus vertices),
			\item report the \emph{ratio} between $w$ and $a$ as the throughput metric.
		\end{enumerate}
 		In other words, instead of averaging the rate per sample, that is, $\textit{EVPS}_i$ for sample $i$, \toolname{} first averages the performance metric and then reports the rate. \futureinversion{2.0}{Maybe change ``rate'' to ``ratio'' in this sentence [Gabor]}
	\item For \textbf{Cost metrics,} the \emph{harmonic mean} across all data. This is because the denominator (e.g., EVPS for PPP) gives meaning to the ratio (TCO is constant across experiments with the same System Under Test), which indicates that the arithmetic mean would be misleading~\cite[S.3.1.1]{DBLP:conf/sc/HoeflerB15}.
	\item For \textbf{Ratio metrics} such as Speedup, the \emph{geometric mean} across all data.
\end{itemize}


\futureinversion{1.0}{In many ranking approaches, it is convenient to have each participant compared not against each other, but against a {\it baseline} (or golden standard, or reference). Among the approaches, we have in particular considered an adapted version of the method of Fractional Difference Comparison~\cite{DBLP:conf/wosp/IlyushkinAHPGEI17}. In this method, a baseline constructed analytically, by ascribing it, for each metric, the best value observed among all platforms in the competition; in other words, the baseline is possibly sub-optimal, but could presumably be achievable in practice, by combining the best features of the participating platforms.}






\subsection{Tournament-based Approach} \label{sec:competitions:tournament}
In a tournament-based approach, the system performance is ranked by means of competitive tournaments~\cite{Thurstone1927}.  Generally, a Round-Robin pair-wise tournament~\cite{David1960} (from hereon, {\it tournament}) of $p$ participants involves a balanced set of (pair-wise) comparisons between the results of each pair of participants; if there are $c$ criteria to compare the participants, there will be $\frac{1}{2} \times c \times p (p - 1)$ pair-wise comparisons. In a pair-wise comparison, a pre-defined amount of points (often, 1 or 3) is given to the better ({\it winner}) participant from the pair. It is also common to give zero points to the worse ({\it loser}) participant from the pair, and to split the points between participants with equal performance. Similar tournaments have been used for decades in chess competitions, in professional sports leagues such as (European and American) football, etc.

We do not consider here other pair-wise tournaments, such as replicated tournaments~\cite{David1960} and unbalanced comparisons~\cite{david1987ranking}, which have been used especially in settings where comparisons are made by human referees and are typically discretized on 5-point Likert scales, and thus are quantitatively less accurate than the \toolname{} measurements.









